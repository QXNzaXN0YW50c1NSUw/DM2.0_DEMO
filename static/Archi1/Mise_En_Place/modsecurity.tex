\section{Mise en place d'apache Modsecurity}

\subsection{Objectifs}

L'objectif est de mettre en place un pare-feu applicatif permettant de protéger les applications de INFRA01 ainsi que de filtrer le mot clef \textit{EPITA} sur les requêtes \textit{POST}, \textit{PUT} et \textit{GET}. Pour cela nous utiliserons ModSecurity \footnote{ModSecurity est un module qui s’intègre au serveur Web Apache et qui permet de mettre en place un pare-feu applicatif dédié aux applications Web. Il fournit un moteur permettant de détecter et de se prémunir des attaques avant qu’elles n’atteignent l’application Web protégée. \\
ModSecurity offre les mécanismes suivants : 
\begin{itemize}
\item la journalisation du trafic ;
\item la surveillance du trafic et détection en temps réel des attaques ;
\item la prévention des attaques et adaptation des règles de détection en vue de corriger les vulnérabilités applicatives sans modifier réellement les applications Web.
\end{itemize}
\cite{modsecurity}} avec Apache2. \\



Le pare-feu présente cependant des limites s'il est utilisé seul. \\
Pour palier ce problème, nous utiliserons OWASP Core Set Rule \footnote{OWASP ModSecurity Core Rule Set (CRS) est un ensemble de règles de détection d'attaques génériques à utiliser avec, entre autres, ModSecurity. Le CRS a pour but de protéger les applications contre une grande variété d'attaques, incluant le top 10 des attaques recensées par OWASP. De plus, il génère très peu de fausses alertes - \url{https://modsecurity.org/crs/}}.

La figure \ref{Apache/modsecurity} montre le fonctionnement de ModSecurity. Cela nous permettra de comprendre comment écrire les règles adéquates.
\begin{figure}[h!]
	\begin{center}
		\includegraphics[scale=0.9]{Apache/modsecurity.jpg}
		\caption{Phases de traitement de l'information par modsecurity \cite{modsecurity-phases}}
		\label{Apache/modsecurity}
	\end{center}
\end{figure}
\FloatBarrier 
Le cercle représente le processus de traitement de données d'Apache. Les étiquettes sont les phases avec lesquelles on pourra interférer via ModSecurity. \\
Ce qu'il faut comprendre :
\begin{itemize}
    \item \textbf{Phase 1} : cette phase se déroule juste après qu'Apache ait fini de lire l'en-tête d'une requête. Il est possible d'influer sur la façon dont la requête va être traitée.
    \item \textbf{Phase 2} : lors de cette phase, le corps de la requête est accessible, il est donc possible d'en obtenir les arguments.
    \item \textbf{Phase 3} : cette phase se déroule juste avant que le service envoie les en-têtes de la réponse au client. Il est donc possible d'acquérir ce qui a été envoyé au client avant qu'il ne le reçoive.
    \item \textbf{Phase 4} : dans cette phase, le corps de la réponse est accessible. Il est possible de détecter des "information disclosure".
    \item \textbf{Phase 5} : phase avant l'entrée d'information dans les logs. Il est possible d'inspecter les messages d'erreurs d'apache.
\end{itemize}

\subsection{Configuration du serveur Apache avec modsecurity}
\subsubsection{Configuration par défaut}
Pour pouvoir utiliser modsecurity avec apache il faut : 
\begin{itemize}
    \item Installer libapache2-modsecurity :
    \begin{minted}{shell}
    $ sudo apt install libapache2-modsecurity
    \end{minted}
    
    \item Activer le module security2 :
    \begin{minted}{shell}
    $ sudo a2enmod security2
    \end{minted}
    
    \item Changer \textbf{/etc/modsecurity/modsecurity-default.conf} en :\\  \textbf{/etc/modsecurity/modsecurity.conf} :
    \begin{minted}{shell}
    $ sudo cp /etc/modsecurity/modsecurity-default.conf \ 
        /etc/modsecurity/modsecurity.conf
    \end{minted}
    
    \item Changer dans le fichier; \textbf{SecRuleEngine DetectionOnly} en :
    \begin{itemize}
        \item \textbf{SecRuleEngine On}
    \end{itemize}
    \item Redémarrer apache :
    \begin{minted}{shell}
    $ sudo systemctl restart apache2
    \end{minted}
    
\end{itemize}
\subsubsection{Changement de la configuration par défaut}
Certains paramètres sont à modifier, on obtiendra  :
\begin{itemize}
    \item \textbf{SecResponseBodyLimitation Reject} : par défaut laisse passer les réponses qui ne peuvent pas être inspectées dû à leur taille.
    \item \textbf{SecTmpDir /var/modsecurity/tmp/} :  par défaut, écrit les fichiers temporaires dans un endroit accessible à tout le monde (/tmp)
    \item \textbf{SecDataDir /var/modsecurity/persistant/} : par défaut, écrit les fichiers persistants dans un endroit accessible à tout le monde (/tmp)
    \item \textbf{SecAuditEngine On} : par défaut ne log que certaines transactions.
\end{itemize}


Il faut ensuite :
\begin{itemize}
    \item Créer les dossiers adéquats :
\begin{minted}{shell}
    $ mkdir -p /var/modsecurity/tmp /var/modsecurity/persistant
\end{minted}
    \item Redémarrer apache2
\end{itemize}

\subsubsection{Filtrage du mot \textbf{EPITA}}
    On cherche à bloquer le mot clé \texttt{EPITA} dans les méthodes \textit{GET}, \textit{PUT} et \textit{POST}. Pour cela il faut rajouter la ligne suivante dans \textbf{/etc/modsecurity/modsecurity.conf} :
    \begin{minted}{shell}
    SecRule REQUEST_METHOD "(?:POST|GET|PUT)" \ 
        "phase:2, t:none, chain, deny, msg:'STOP', status:500, id:5555"
    SecRule REQUEST_URI|ARGS|REQUEST_BODY "EPITA" \ 
        "log, auditlog, tag:'KeyWord_EPITA_matched'"
    \end{minted}
    On regarde si \textit{EPITA} est présent après avoir regardé si on recevait une requête GET/POST/PUT. Le message d'erreur sera le 500 et l'évènement sera placé dans les logs. Tout se passe pendant la phase 2.\\ 
    
    Redémarrer apache :
    \begin{minted}{shell}
    $ sudo systemctl restart apache2.
    \end{minted}
    
\subsubsection{OWASP ModSecurity Core Rule Set}

La Configuration par défaut ne suffisant pas, il est possible de se protéger en utilisant le CRS d'OWASP.
OWASP CSR est à ce jour disponible par défaut avec l'installation de libapache2-mod-security.

Pour installer OWASP ModSecurity CRS manuellement il faut : 
\begin{itemize}
    \item Cloner le dépôt du CRS :
    \begin{minted}{shell}
    $ git clone https://github.com/SpiderLabs/owasp-modsecurity-crs.git
    \end{minted}
    \item Inclure \textbf{crs-setup.conf} et \textbf{rules/*.conf} dans la configuration : 
    \begin{itemize}
        \item \textbf{/etc/apache2/mods-enabled/security2\_module}.
    \end{itemize}
    \item Redémarrer apache2.
\end{itemize}