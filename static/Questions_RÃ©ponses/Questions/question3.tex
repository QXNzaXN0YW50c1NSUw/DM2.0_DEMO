\section{Comment le logiciel identifie-t-il les services exposés ?}

Pour Nmap, une fois que les ports TCP et/ou UDP ont été découverts grâce à l'une des méthodes de scan, la détection de version interroge ces ports-là pour en savoir plus sur le type de service qui tourne dessus. La base de données \textit{nmap-service-probes} contient des sondes pour interroger de multiples services, et comparer les expressions pour reconnaître les réponses et les analyser. Nmap essaye de déterminer le protocole du service, le nom de l'application, le numéro de version, le nom d'hôte, le type de périphérique, et la famille du système d'exploitation. \\
\indent Lorsque des services RPC ont été mis en évidence, le module RPC de Nmap est utilisé pour déterminer le programme RPC et les numéros de version. Il prend tous les ports TCP/UDP détectés comme étant du RPC, et les inonde avec des commandes NULL du programme \textit{SunRPC}, pour tenter de déterminer s'il s'agit bien de ports RPC, et si oui, de déterminer quel programme et quel numéro de version ils servent. \\
\indent Lorsque Nmap reçoit des réponses d'un service mais qu'il ne peut pas les faire correspondre à sa base de données, il affiche une empreinte spéciale ainsi qu'une URL à laquelle il est possible de soumettre ce qui s'exécute sur ce port ; mais cela uniquement si l'on est certain de ce qui s'exécute.

\begin{comment}
After TCP and/or UDP ports are discovered using one of the other scan methods, version detection interrogates those ports to determine more about what is actually running. The nmap-service-probes database contains probes for querying various services and match expressions to recognize and parse responses. Nmap tries to determine the service protocol (e.g. FTP, SSH, Telnet, HTTP), the application name (e.g. ISC BIND, Apache httpd, Solaris telnetd), the version number, hostname, device type (e.g. printer, router), the OS family (e.g. Windows, Linux).

When RPC services are discovered, the Nmap RPC grinder is automatically used to determine the RPC program and version numbers. It takes all the TCP/UDP ports detected as RPC and floods them with SunRPC program NULL commands in an attempt to determine whether they are RPC ports, and if so, what program and version number they serve up. Thus you can effectively obtain the same info as rpcinfo -p even if the target's portmapper is behind a firewall (or protected by TCP wrappers). Decoys do not currently work with RPC scan.

When Nmap receives responses from a service but cannot match them to its database, it prints out a special fingerprint and a URL for you to submit if to if you know for sure what is running on the port. Please take a couple minutes to make the submission so that your find can benefit everyone. Thanks to these submissions, Nmap has about 6,500 pattern matches for more than 650 protocols such as SMTP, FTP, HTTP, etc.
\end{comment}